\section{Introdução}
\subsection{Contextualização e Justificativa}
A utilização de dispositivos móveis tem se tornado algo habitual. A cada dia que passa, surgem aparelhos com novos recursos e melhorias, deixando não só os dispositivos antigos como as técnicas de segurança muitas vezes obsoletos. Segundo \cite{FAST}, as pessoas utilizam mais o celular do que o computador pessoal, seja por motivos de trabalho ou entretenimento. Segundo a mesma pesquisa, em média, os donos de smartphones possuem 15 aplicativos instalados. Esse crescimento é uma ótima oportunidade para desenvolvedores entrarem em um mercado novo e que tem bastante potencial, considerando que estatisticamente o mercado cresce bem mais do que o de computadores pessoais. A utilização de dispositivos móveis vem crescendo de forma impressionante e a preocupação com segurança da informação deve acompanhar essa evolução, e tanto desenvolvedores quanto usuários podem contribuir com isso. Segundo \cite{FAST}, o tempo gasto on-line em um dispositivo móvel triplicou entre os anos de 2012 e 2015, os aplicativos presentes nesses dispositivos podem facilitar bastante a vida dos usuários, mas também podem representar um risco. Informações pessoais, dados de lugares visitados, telefones, mensagens, etc, o smartphone é uma fonte de informações prontas para serem roubadas. Conhecer todos os recursos, limitações e capacidades disponíveis é essencial para melhorar aspectos de segurança da informação. 

\subsection{Objetivos}
\subsubsection{Objetivos Gerais}
Dentro deste contexto, o presente trabalho apresenta boas práticas de programação buscando a mitigação de problemas conhecidos relacionados a segurança da informação em dispositivos móveis, com foco no sistema operacional iOS. Buscando os insumos necessários para a criação de um aplicativo do tipo Quiz, que teste e ao mesmo tempo incentive a adoção dessas boas práticas durante o desenvolvimento de aplicações. Para o cumprimento do objetivo geral, foram traçados alguns objetivos específicos citados abaixo.

\subsubsection{Objetivos Específicos}
\begin{enumerate}
    \item Levantamento sobre a história da computação móvel e sua evolução buscando entender a dinâmica relacionada à segurança da informação;
    \item Levantamento sobre as tecnologias e recursos existentes em dispositivos móveis a fim de encontrar quais as limitações e vulnerabilidades de cada tecnologia;
    \item Levantamento sobre a história dos dispositivos Apple e como as aplicações se relacionam tanto com o sistema operacional quanto com outras aplicações;
    \item Estudo de problemas e vulnerabilidades conhecidas, mesmo que em outras plataformas, buscando aprender com problemas antigos mas que não deixaram de representar um risco;
    \item Levantamento de boas práticas comumente utilizadas em desenvolvimento seguro para qualquer plataforma, porém com foco na parte mobile;
    \item Apresentaçao dessas boas práticas de forma lúdica em um aplicativo do tipo “Quiz” com o objetivo de testar as habilidades dos desenvolvedores no que diz respeito à utilização de técnicas que visam segurança.

\end{enumerate}

\begin{figure}[h]
\label{figura1}
\centering
\includegraphics[scale=0.4]{secoes/figuras/img.jpg}
\caption{Security}
\end{figure}